\documentclass[compilation.tex]{subfiles}

\begin{document}

\section{Summary}
\label{sec:summary}

\section{Introduction}
\label{sec:introduction}

\section{Part 1: Download application}
\label{sec:downloadapp}

\section{Part 2: Network}
\label{sec:network}

\subsection{Exp 1: Configure an IP Network}
\label{exp:1}

% cables:
%   $ tux31@eth0 -- tux34@eth0
% tux31:
%   # ifconfig eth0 up
%   # ifconfig eth0 172.16.30.1/24
%   # route add default gw 172.16.30.254
% tux34:
%   # ifconfig eth0 up
%   # ifconfig eth0 172.16.30.254/24
% >1:
%   ARP packets are sent in a network to request the MAC address
%   of the machine with a certain IPv4 address. The protocol specifies
%   the packet is sent to every host (machine) on the network, and the
%   one with that IPv4 address responds with its MAC address.
%   ARP: convert 32-bit logical IP address to a 48-bit physical MAC address.
% >2:
%   tux31@eth0 sends an ARP request for 172.16.30.254 on 172.16.30.0/24, which
%   reaches tux34@eth0 (the only other host on the network). It sends an ARP
%   response of 00:21:5a:5a:7d:74 back to 172.16.30.1 (tux41@eth0).
% >3:
%   ICMP uses IPv4, protocol number: 1
%   ICMP ECHO_REQUEST
%    ICMP type: 8, Code: 0
% >4:
%   IP packet of ICMP ECHO_REQUEST:
%      Source IP is 172.16.30.1
%      Destination IP is 172.16.30.254
%      Reverse for ICMP ECHO_REPLY
%   Ethernet frame wrapper of ICMP ECHO_REQUEST:
%      Source MAC address is 00:0f:fe:8b:e4:4d
%      Destination MAC address is 00:21:5a:5a:7d:74
%      Reverse for wrapper of ICMP ECHO_REPLY
%   ICMP ECHO_REPLY
%    ICMP type: 0, Code: 0
% >5:
%   Check the EtherType (in wireshark: eth.type, bytes 12-13):
%     0x0800 --> IPv4 protocol
%                then check the protocol in the IP header
%                1 --> ICMP
%                6 --> TCP
%               17 --> UDP
%     0x0806 --> ARP
%     0x8035 --> RARP
% >6:
%   Ethernet frame: No header field specifying length.
%     Read the whole thing and measure.
%   IPv4 packet: the header has two length fields, one for header
%     length (4 bits, offset 4 bits) and another for packet length
%     (2 bytes, offset 16 bits)
% >7:
%   172.0.0.0/8 is a range of 2²⁴ IPv4 addresses representing the host
%   machine. Useful for testing or running client-server services in the
%   host. Generally, in linux, only 172.0.0.1 is used and it is assigned
%   the local loopback interface lo, though it is not a rule.

\subsection{Exp 2: Implement two virtual LANs in a switch}
\label{exp:2}

% cables:
%   $ tux31@eth0 -- sw Fa0/1
%   $ tux34@eth0 -- sw Fa0/4
%   $ tux32@eth0 -- sw Fa0/2
% tux42:
%   # ifconfig eth0 up
%   # ifconfig eth0 172.16.31.1/24
% >1:
%   create vlan 30 and vlan 31
%   add ports Fa0/1 and Fa0/4 to vlan 30 and port Fa0/2 to vlan 31.
% >2:
%   two broadcast domains (networks), 172.16.30.0/24 for vlan 30,
%   and 172.16.31.0/24 for vlan 31.

\subsection{Exp 3: Configure a Router in Linux}
\label{exp:3}

% cables:
%   $ tux31@eth0 -- sw Fa0/1
%   $ tux34@eth0 -- sw Fa0/4
%   $ tux34@eth1 -- sw Fa0/7
%   $ tux32@eth0 -- sw Fa0/2
% tux44:
%   # ifconfig eth1 up
%   # ifconfig eth1 172.16.31.253/24
%   # echo 1 > /proc/sys/net/ipv4/ip_forward
%   # echo 0 > /proc/sys/net/ipv4/icmp_echo_ignore_broadcasts
% tux42:
%   # route add -net 172.16.30.0/24 gw 172.16.31.253
% >1:
%   Destination    Gateway        Mask           Interface
%   -- routes in tux31:
%   0.0.0.0        172.16.30.254  0.0.0.0        eth0
%   default via 172.16.30.254 eth0
%     default gateway of tux31. If no other route matches the destination
%     of an IP packet, send it to 172.16.30.254 (through interface eth0).
%   172.16.30.0    0.0.0.0        255.255.255.0  eth0
%   172.16.30.0/24 dev eth0
%     means that tux31 is directly connected to all hosts with addresses
%     172.16.30.0 with mask 255.255.255.0 (that is, network
%     172.16.30.0/24) through interface eth0.
%   -- routes in tux34:
%   172.16.30.0    0.0.0.0        255.255.255.0  eth0
%   172.16.30.0/24 dev eth0
%     means that tux34 is directly connected to network 172.16.30.0/24
%     through interface eth0.
%   172.16.31.0    0.0.0.0        255.255.255.0  eth1
%   172.16.31.0/24 dev eth1
%     means that tux34 is directly connected to network 172.16.31.0/24
%     through interface eth1.
%   -- routes in tux32:
%   172.16.30.0    172.16.31.253  255.255.255.0  eth0
%   172.16.30.0/24 via 172.16.31.253 dev eth0
%     means that tux32 is connected to all hosts in network 172.16.30.0/24
%     through gateway 172.16.31.253. So any packet whose destination is in network
%     172.16.30.0/24 should be sent to 172.16.31.253, which will forward it.
%   172.16.31.0    0.0.0.0        255.255.255.0  eth0
%   172.16.31.0/24 dev eth0
%     means that tux32 is directly connected to network 172.16.31.0/24
%     through interface eth0.
% >2:
%   Metric:
%     Scores the route. It can contain any number of values that help a router
%     or host determine the best route among multiple routes to a destination.
%     A packet will generally be sent through the route with the lowest metric.
%     The most basic metric is typically based on path length, hop count or delay.
%   Flags (route -n):
%     U - Up, means the route is up.
%     G - Gateway, means that the route is to a gateway.
%         Without this flag present, it means the route is directly connected.
%     H - Host, means the route is to a host, the destination is a complete host address.
%         For route -n, this is equivalent to mask 255.255.255.255
%     D - reDirected, means this route was created by a redirect.
%     M - Modified, means this route was modified by a redirect.
%   Ref: number of references to this route (not used in the linux kernel)
%   Iface: interface to be used for this route.
% >3:
%   ARP Request/Reply pairs which are required for communication of ICMP:
%     tux31 (172.16.30.1) asks MAC address of 172.16.30.254 (tux34)
%     tux34 (172.16.31.253) asks MAC address of 172.16.31.1 (tux32)
%   Others:
%     tux34 (172.16.30.254) asks MAC address of 172.16.30.1 (tux31)
%     tux32 (172.16.31.1) asks MAC address of 172.16.31.253 (tux34)
%   tux31@eth0 (172.16.30.1): 00:0f:fe:8b:e4:4d
%   tux34@eth0 (172.16.30.254): 00:21:5a:5a:7d:74
%   tux34@eth2 (172.16.31.253): 00:01:02:21:83:0e
%   tux32@eth0 (172.16.31.1): 00:21:5a:61:30:63
% >4:
%   ICMP ECHO_REQUEST and ICMP ECHO_REPLY
%   Ping requests are ECHO_REQUEST and replies are ECHO_REPLY.
%   tux31 creates requests and accepts replies.
%   tux32 creates replies and accepts requests.
%   tux34 forwards everything.
% >5:
%   When tux31 pings tux32:
%     ECHO_REQUEST:
%     The IP addresses are fixed (source: 172.16.30.1, destination 172.16.31.1)
%     but the MAC addresses reflect the hops the packet takes through the network.
%     From tux31@eth0 to tux34@eth0, source MAC address is that of tux31@eth0 and
%     destination MAC address is that of tux34@eth0. From tux34@eth2 to tux32@eth0
%     source MAC address is that of tux34@eth2 and destination MAC address is that
%     of tux32@eth0.
%     For ECHO_REPLY it's the reverse

\subsection{Exp 4: Configure a Commercial Router and Implement NAT}
\label{exp:4}

\subsection{Exp 5: DNS}
\label{exp:5}

\subsection{Exp 6: TCP connections}
\label{exp:6}

\section{Conclusion}
\label{sec:conclusion}

\section{References}
\label{sec:references}

\end{document}
