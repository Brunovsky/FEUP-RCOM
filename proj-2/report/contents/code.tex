\documentclass[subfiles]{main.tex}

\begin{document}

\section{Appendix: Code}

\section*{pipeline.c}

\begin{lstlisting}[style=C-sublime]
#include "pipeline.h"
#include "debug.h"

#include <stdlib.h>
#include <unistd.h>
#include <string.h>
#include <stdio.h>
#include <regex.h>

#include <netdb.h>
#include <sys/socket.h>
#include <netinet/in.h>
#include <arpa/inet.h>

//
// FTP Control Socket (User-Protocol Interpreter)
//
static FILE* controlstream = NULL;
static int controlfd = 0;

//
// FTP Passive Socket (Data Transfer)
//
static FILE* passivestream = NULL;
static int passivefd = 0;

//
// Input FTP URL
//
static url_t url;

static void close_control_socket();

static void close_passive_socket();

static void free_url();

//
// Send an FTP command to the control socket.
//
static int send_ftp_command(const char* command) {
	ssize_t count = 0, s = 0;
	ssize_t len = strlen(command);
	
	logftpcommand(command);
	
	do {
	count += s = send(controlfd, command + count, len - count, 0);
	if (s <= 0) libfail("Failed to write anything to control socket");
	} while (count < len);
	
	return 0;
}

//
// Read and discard reply lines from the FTP control socket stream until reading a
// line that starts with an FTP code not followed by a dash.
//
static int recv_ftp_reply(char* code, char** line) {
	char* reply = NULL;
	
	ssize_t read_size = 0;
	
	do {
	free(reply);
	
	reply = NULL;
	size_t n = 0;
	
	read_size = getline(&reply, &n, controlstream); // keeps \r\n
	if (read_size < 1) libfail("Failed to read reply from control socket stream");
	
	logftpreply(reply);
	} while ('1' > reply[0] || reply[0] > '5' || read_size < 4 || reply[3] == '-');
	
	strncpy(code, reply, 3); code[3] = '\0';
	line != NULL ? *line = reply : free(reply);
	
	return 0;
}

//
// Extract substring matched by a capture group from the source string,
// returning NULL if the capture group didn't match.
//
static char* regexcap(const char* source, regmatch_t match) {
	ssize_t start = match.rm_so;
	ssize_t end = match.rm_eo;
	if (start == -1) return NULL;
	
	size_t len = end - start;
	
	char* buffer = malloc((len + 1) * sizeof(char));
	strncpy(buffer, source + start, len);
	buffer[len] = '\0';
	
	return buffer;
}

//
// 1. Parse program's input, the FTP URL
// For URL parsing we use a weak regular expression. It accepts all valid (ftp) URLs
// with non-empty filenames, although it also passes a whole bunch of invalid ones too.
// It gets the job done quickly. Note: it does not accept a port after the host.
//
int parse_url(const char* urlstr) {
	regex_t regex;
	// capturers:     1       23         4 5              6          78          9
	//                 ftp ://[user      [:pass]       @] host      /path/ to/   filename
	regcomp(&regex, "^(ftp)://(([^:@/ ]+)(:([^:@/ ]+))?@)?([^:@/ ]+)/(([^/ ]+/)*)([^/ ]+)$",
	REG_EXTENDED | REG_ICASE | REG_NEWLINE);
	
	regmatch_t pmatch[10];
	
	// Regex-parse the input url
	int s = regexec(&regex, urlstr, 10, pmatch, 0);
	regfree(&regex);
	if (s != 0) fail("Invalid URL {no port, must have nonempty filename}");
	
	// Take captures
	url = (url_t){
	.protocol = regexcap(urlstr, pmatch[1]),
	.username = regexcap(urlstr, pmatch[3]),
	.password = regexcap(urlstr, pmatch[5]),
	.hostname = regexcap(urlstr, pmatch[6]),
	.pathname = regexcap(urlstr, pmatch[7]),
	.filename = regexcap(urlstr, pmatch[9]),
	.port = 21
	};
	
	atexit(free_url);
	
	progress("1. FTP URL: "CGREEN"ftp://%s/%s%s"CEND, url.hostname, url.pathname, url.filename);
	
	// Pick default username
	if (url.username == NULL) {
	url.username = strdup("anonymous"); // *** DEFAULT USER
	progress(" Defaulting username to %s", url.username);
	}
	
	// Pick default password
	if (url.password == NULL) {
	url.password = strdup("upstudent-rcom"); // *** DEFAULT PASS
	progress(" Defaulting password to %s", url.password);
	}
	
	// Extra: warning for the common test case where host is ftp.up.pt
	if (strcmp(url.username, "anonymous") && !strcmp(url.hostname, "ftp.up.pt")) {
	printf(CYELLOW"\n Warning: ftp.up.pt operates only in anonymous mode.\n\n"CEND);
	}
	
	progress(" url.protocol: %s", url.protocol);
	progress(" url.username: %s", url.username);
	progress(" url.password: %s", url.password);
	progress(" url.hostname: %s", url.hostname);
	progress(" url.pathname: %s", url.pathname);
	progress(" url.filename: %s", url.filename);
	progress(" url.port: %d", url.port);
	
	return 0;
}

//
// 2. Resolve hostname to server's IPv4 address and setup control socket
//
int ftp_open_control_socket() {
	char code[4];
	
	progress("2. Resolve hostname %s and setup control socket", url.hostname);
	
	// 2.1. resolve
	struct hostent* host = gethostbyname(url.hostname);
	if (host == NULL) libfail("Could not resolve hostname %s", url.hostname);
	
	progress(" 2.1. Resolved hostname %s successfully", url.hostname);
	progress("  host->h_name: %s", host->h_name);
	progress("  host->h_addrtype: %d [IPv4=%d]", host->h_addrtype, AF_INET);
	
	// 2.2. extract IPv4 address
	if (host->h_addrtype != AF_INET) fail("Expected IPv4 address");
	
	char* protocolip = inet_ntoa(*(struct in_addr*)host->h_addr);
	
	progress(" 2.2. Protocol IP Address: %s", protocolip);
	progress("      Establishing control connection with FTP server");
	
	// 2.3. socket()
	controlfd = socket(AF_INET, SOCK_STREAM, 0);
	if (controlfd == -1) libfail("Failed to open socket for control connection");
	
	controlstream = fdopen(controlfd, "r");
	atexit(close_control_socket);
	
	progress(" 2.3. Opened control socket for FTP's protocol connection");
	
	// 2.4. connect()
	struct sockaddr_in in_addr = {0};
	in_addr.sin_family = AF_INET; // IPv4 address
	in_addr.sin_port = htons(url.port); // host byte order -> network byte order
	in_addr.sin_addr.s_addr = inet_addr(protocolip); // to network format
	
	int s = connect(controlfd, (struct sockaddr*)&in_addr, sizeof(in_addr));
	if (s != 0) libfail("Failed to connect() control socket");
	
	progress(" 2.4. Connected control socket to %s:%d", protocolip, 21);
	
	// 2.5. recv() 220 === Service ready for new user
	recv_ftp_reply(code, NULL);
	if (strcmp(code, "220") != 0) unexpected("Expected reply 220, got %s", code);
	
	progress(" 2.5. Successfully established control socket connection");
	
	return 0;
}

//
// 3. Login user.
// Simply a USER command followed by a PASS command.
//
int ftp_login() {
	char code[4];
	
	progress("3. Send USER and PASS login commands to FTP server");
	
	// 3.1. send() USER username
	//      recv() 331 === User name okay, send password
	char* user_command = malloc((10 + strlen(url.username)) * sizeof(char));
	sprintf(user_command, "USER %s\r\n", url.username);
	
	send_ftp_command(user_command);
	free(user_command);
	
	recv_ftp_reply(code, NULL);
	if (strcmp(code, "331") != 0) unexpected("Expected reply 331, got %s", code);
	
	progress(" 3.1. Server confirmed user %s", url.username);
	
	// 3.2. send() PASS password
	//      recv() 230 === Login successful, proceed
	char* pass_command = malloc((10 + strlen(url.password)) * sizeof(char));
	sprintf(pass_command, "PASS %s\r\n", url.password);
	
	send_ftp_command(pass_command);
	free(pass_command);
	
	recv_ftp_reply(code, NULL);
	if (strcmp(code, "230") != 0) unexpected("Expected reply 230, got %s", code);
	
	progress(" 3.2. Server acknowledges login, proceeding");
	
	return 0;
}

//
// 4. Enter passive mode
//
int ftp_open_passive_socket() {
	char code[4];
	
	progress("4. Establish passive connection with FTP server");
	
	// 4.1. send() PASV
	//      recv() 227 Entering Passive Mode (IP3.IP2.IP1.IP0,Port1,Port0)
	send_ftp_command("PASV \r\n");
	
	char* line;
	recv_ftp_reply(code, &line);
	if (strcmp(code, "227") != 0) unexpected("Expected reply 227, got %d", code);
	
	regex_t regex;
	regcomp(&regex, "([0-9]+, ?[0-9]+, ?[0-9]+, ?[0-9]+, ?[0-9]+, ?[0-9]+)",
	REG_EXTENDED | REG_ICASE);
	
	regmatch_t pmatch[2];
	
	int s = regexec(&regex, line, 2, pmatch, 0);
	regfree(&regex);
	if (s != 0) fail("Unexpected Passive Mode response format: %s", line);
	
	char* substr = regexcap(line, pmatch[1]);
	free(line);
	
	progress(" 4.1. Entered Passive Mode: (%s)", substr);
	
	// 4.2. extract passive ip from response line
	int ip3, ip2, ip1, ip0, port1, port0;
	sscanf(substr, "%d, %d, %d, %d, %d, %d", &ip3, &ip2, &ip1, &ip0, &port1, &port0);
	free(substr);
	
	char passiveip[20] = {0};
	sprintf(passiveip, "%d.%d.%d.%d", ip3, ip2, ip1, ip0);
	int port = port1 * 256 + port0;
	
	progress(" 4.2. Passive IP Address: %s:%d", passiveip, port);
	
	// 4.3. socket()
	passivefd = socket(AF_INET, SOCK_STREAM, 0);
	if (passivefd == -1) libfail("Failed to open socket for passive connection");
	
	passivestream = fdopen(passivefd, "r+");
	atexit(close_passive_socket);
	
	progress(" 4.3. Opened passive socket for FTP's passive connection");
	
	// 4.4. connect()
	struct sockaddr_in in_addr = {0};
	in_addr.sin_family = AF_INET; // IPv4 address
	in_addr.sin_port = htons(port); // host bytes -> network bytes
	in_addr.sin_addr.s_addr = inet_addr(passiveip); // to network format
	
	s = connect(passivefd, (struct sockaddr*)&in_addr, sizeof(in_addr));
	if (s != 0) libfail("Failed to connect() passive socket");
	
	progress(" 4.4. Connected passive socket to %s:%d", passiveip, port);
	progress(" 4.5. Successfully established passive socket connection");
	
	return 0;
}

//
// 5. Send Retrieve command to FTP server
// This command instructs the server to send the filename through the passive connection.
//
int send_retrieve() {
	char code[4];
	
	progress("5. Retrieve file from Server (retrieve command)");
	
	// 5.1. send() RETR filepath
	//      recv() 150 File status okay, opening data transfer now
	size_t len = strlen(url.pathname) + strlen(url.filename);
	char* retr_command = malloc((10 + len) * sizeof(char));
	sprintf(retr_command, "RETR %s%s\r\n", url.pathname, url.filename);
	
	send_ftp_command(retr_command);
	free(retr_command);
	
	recv_ftp_reply(code, NULL);
	if (strcmp(code, "150") != 0) unexpected("Expected reply 150, got %s", code);
	
	progress(" 5.1. Confirmed, server retrieved %s%s", url.pathname, url.filename);
	
	return 0;
}

//
// 6. Download file retrieved.
// It is being sent through TCP to port 20, we just read the socket
// and forward to the output file stream until the transmission is over.
//
int download_file() {
	char code[4];
	
	progress("6. Download file %s into current directory", url.filename);
	
	// 6.1. Open output file in current directory
	FILE* out = fopen(url.filename, "w"); // streams are love, streams are life
	if (out == NULL) libfail("Failed to open output file in current directory");
	
	progress(" 6.1. Opened output file successfully");
	
	// 6.2. Canonical reading loop on passivefd, read the whole thing
	char buffer[1024]; // TCP caps at 1500B/p practically
	ssize_t read_size = 0;
	
	progress("      Reading...");
	
	while ((read_size = read(passivefd, buffer, sizeof(buffer))) > 0) {
	size_t write_size = fwrite(buffer, read_size, 1, out);
	if (write_size != 1) libfail("Failed to write to output file");
	}
	
	if (read_size < 0) libfail("Failed to read file on passive socket");
	
	fclose(out);
	
	progress(" 6.2. "CGREEN"Done."CEND);
	
	// 6.3. recv() 226 === Closing data connection, transfer complete
	recv_ftp_reply(code, NULL);
	if (strcmp(code, "226") != 0) unexpected("Expected reply 226, got %d", code);
	
	return 0;
}

//
// 7. Close the connection to the server.
//
static void close_control_socket() {
	send_ftp_command("QUIT \r\n");
fclose(controlstream);
}

static void close_passive_socket() {
	fclose(passivestream);
}

static void free_url() {
	free(url.protocol);
	free(url.username);
	free(url.password);
	free(url.hostname);
	free(url.pathname);
	free(url.filename);
}
\end{lstlisting}

\section*{pipeline.h}

\begin{lstlisting}[style=C-sublime]
#ifndef PIPELINE_H___
#define PIPELINE_H___

typedef struct {
	char* protocol;
	char* username;
	char* password;
	char* hostname;
	char* pathname;
	char* filename;
	int port;
} url_t;

int parse_url(const char* urlstr);

int ftp_open_control_socket();

int ftp_login();

int ftp_open_passive_socket();

int send_retrieve();

int download_file();

#endif // PIPELINE_H___
\end{lstlisting}

\section*{main.c}

\begin{lstlisting}[style=C-sublime]
#include "pipeline.h"

#include <stdlib.h>
#include <stdio.h>

int main(int argc, char** argv) {
	if (argc != 2) {
		printf("Expected 1 argument. Usage:\n  ./download ftp://...\n");
		return 0;
	}
	
	// 1. Parse input FTP URL
	parse_url(argv[1]);
	
	// 2. Resolve hostname to server's IPv4 IP address and open protocol socket
	ftp_open_control_socket();
	
	// 3. Login to the server (user + password)
	ftp_login();
	
	// 4. Enter passive mode and open passive socket
	ftp_open_passive_socket();
	
	// 5. Send retrieve command for file
	send_retrieve();
	
	// 6. Download file
	download_file();
	
	// 7. Close connection to server
	exit(EXIT_SUCCESS);
}
\end{lstlisting}

\section*{debug.c}

\begin{lstlisting}[style=C-sublime]
#include "debug.h"

#include <stdio.h>
#include <stdlib.h>
#include <errno.h>

void progress(const char* format, ...) {
	va_list arglist;
	va_start(arglist, format);
	#if PRINT_PROGRESS
	vprintf(format, arglist);
	printf("\n");
	#endif
	va_end(arglist);
}

void logftpcommand(const char* line) {
	#if PRINT_FTP_COMMAND
	printf(CBLUE"    [COMMD] %s"CEND, line);
	#endif
}

void logftpreply(const char* line) {
	#if PRINT_FTP_REPLY
	printf(CPURP"    [REPLY] %s"CEND, line);
	#endif
}

void fail(const char* format, ...) {
	printf(CRED"FAIL: ");
	va_list arglist;
	va_start(arglist, format);
	vfprintf(stderr, format, arglist);
	va_end(arglist);
	printf("\n"CEND);
	exit(EXIT_FAILURE);
}

void libfail(const char* format, ...) {
	int err = errno;
	printf(CRED"ERROR: ");
	va_list arglist;
	va_start(arglist, format);
	vfprintf(stderr, format, arglist);
	va_end(arglist);
	printf("\n");
	errno = err;
	perror("Library error (errno)");
	printf(CEND);
	exit(EXIT_FAILURE);
}

void unexpected(const char* format, ...) {
	printf(CRED"UNEXPECTED SERVER RESPONSE: ");
	va_list arglist;
	va_start(arglist, format);
	vfprintf(stderr, format, arglist);
	va_end(arglist);
	printf("\n"CEND);
	exit(EXIT_FAILURE);
}
\end{lstlisting}

\section*{debug.h}

\begin{lstlisting}[style=C-sublime]
#ifndef DEBUG_H___
#define DEBUG_H___

#include <stdarg.h>

//
// Linux terminal colors
//
#define CRED "\e[1;31m"

#define CGREEN "\e[1;32m"

#define CYELLOW "\e[1;33m"

#define CBLUE "\e[1;34m"

#define CPURP "\e[1;35m"

#define CEND "\e[m"

//
// Print what?
//
#define PRINT_PROGRESS 1

#define PRINT_FTP_COMMAND 1

#define PRINT_FTP_REPLY 1

//
// Logging functions
//
void progress(const char* format, ...);

void logftpcommand(const char* line);

void logftpreply(const char* line);

void fail(const char* format, ...);

void libfail(const char* format, ...);

void unexpected(const char* format, ...);

#endif // DEBUG_H___
\end{lstlisting}

\end{document}