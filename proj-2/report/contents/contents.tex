\documentclass[compilation.tex]{subfiles}

\begin{document}

\section{Summary}
\label{sec:summary}

\section{Introduction}
\label{sec:introduction}

\section{Part 1: Download application}
\label{sec:downloadapp}

\section{Part 2: Network}
\label{sec:network}

\subsection{Exp 1: Configure an IP Network}
\label{exp:1}

% cables:
%   $ tux41@eth0 -- tux44@eth0
% tux41:
%   # ifconfig eth0 up
%   # ifconfig eth0 172.16.40.1/24
%   # route add default gw 172.16.40.254
% tux44:
%   # ifconfig eth0 up
%   # ifconfig eth0 172.16.40.254/24
% >1:
%   ARP packets are sent in a network to request the MAC address
%   of the machine with a certain IPv4 address. The protocol specifies
%   the packet is sent to every host (machine) on the network, and the
%   one with that IPv4 address responds with its MAC address.
%   ARP: convert 32-bit logical IP address to a 48-bit physical MAC address.
% >2:
%   tux41@eth0 sends an ARP request for 172.16.40.254 on 172.16.40.0/24, which
%   reaches tux44@eth0 (the only other host on the network). It sends an ARP
%   response of 00:21:5a:5a:7d:74 back to 172.16.40.1 (tux41@eth0).
% >3:
%   ICMP ECHO_REQUEST (protocol number = 1)
% >4:
%   IP packet of ICMP ECHO_REQUEST:
%      Source IP is 172.16.40.1
%      Destination IP is 172.16.40.254
%      Reverse for ICMP ECHO_REPLY
%   Ethernet frame wrapper of ICMP ECHO_REQUEST:
%      Source MAC address is 00:0f:fe:8b:e4:4d
%      Destination MAC address is 00:21:5a:5a7d:74
%      Reverse for wrapper of ICMP ECHO_REPLY
% >5:
%   Check the EtherType (in wireshark: eth.type, bytes 12-13):
%     0x0800 --> IPv4 protocol
%                then check the protocol in the IP header
%                1 --> ICMP
%                6 --> TCP
%               17 --> UDP
%     0x0806 --> ARP
%     0x8035 --> RARP
% >6:
%   Ethernet frame: No header field specifying length.
%     Read the whole thing and measure.
%   IPv4 packet: the header has two length fields, one for header
%     length (4 bits, offset 4 bits) and another for packet length
%     (2 bytes, offset 16 bits)
% >7:
%   172.0.0.0/8 is a range of 2²⁴ IPv4 addresses representing the host
%   machine. Useful for testing or running client-server services in the
%   host. Generally, in linux, only 172.0.0.1 is used and it is assigned
%   the local loopback interface lo, though it is not a rule.

\subsection{Exp 2: Implement two virtual LANs in a switch}
\label{exp:2}

% cables:
%   $ tux41@eth0 -- sw Fa0/1
%   $ tux44@eth0 -- sw Fa0/4
%   $ tux42@eth0 -- sw Fa0/2
% tux42:
%   # ifconfig eth0 up
%   # ifconfig eth0 172.16.41.1/24
% >1:
%   create vlan 40 and vlan 41
%   add ports Fa0/1 and Fa0/4 to vlan 40 and port Fa0/2 to vlan 41.
% >2:
%   two broadcast domains (networks), 172.16.40.0/24 for vlan 40,
%   and 172.16.41.0/24 for vlan 41.

\subsection{Exp 3: Configure a Router in Linux}
\label{exp:3}

\subsection{Exp 4: Configure a Commercial Router and Implement NAT}
\label{exp:4}

\subsection{Exp 5: DNS}
\label{exp:5}

\subsection{Exp 6: TCP connections}
\label{exp:6}

\section{Conclusion}
\label{sec:conclusion}

\section{References}
\label{sec:references}

\end{document}
